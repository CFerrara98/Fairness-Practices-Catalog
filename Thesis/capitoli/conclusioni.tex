\chapter{Conclusioni} %\label{1cap:spinta_laterale}
% [titolo ridotto se non ci dovesse stare] {titolo completo}
%

Lo studio è nato con l'intento di chiarire aspetti legati alla fairness e da dove potessero nascere le discriminazioni. Tutto ciò è stato chiarito grazie alla partecipazione di 120 partecipanti circa, dai quali sono state estratte informazioni decisamente utili al punto da chiarire come le aziende si approcciano al problema e quali pratiche potrebbero essere adottate per mitigare quanto possibile le discriminazioni. Ovviamente, tale tesi si concentra su cause e pratiche, quindi molti aspetti ricavati dalle opinioni dei partecipanti sul questionario non sono stati trattati. Tuttavia, le informazioni ricavate inerenti alle pratiche sono tutt'altro che banali dato che è stato reso possibile effettuare deduzioni e nuove teorie, ovviamente orientate sulla base dell'opinione dei partecipanti. Inoltre, aver reso il catalogo delle practice pubblico tramite una wiki concede quella sensazione di "utilità verso gli altri", in modo tale che chiunque abbia a che fare con la fairness abbia un punto di riferimento di cui fidarsi, dato che attualmente, nel web, le risorse correlate sono davvero poche. La wiki rimarrà, quindi, pubblica e aperta a tutti, chiunque potrà analizzare i dati tramite quest'ultima e trarre le proprie conclusioni, le quali potrebbero essere ben diverse dalle implicazioni discusse nell'apposito paragrafo: proprio per questo, chiunque potrà suggerire modifiche e collaborare, per un catalogo che possa essere di aiuto a tutti. D'altra parte, il senso della ricerca dovrebbe essere questo: produrre qualcosa di socialmente utile che possa aiutare gli altri.

\newpage
